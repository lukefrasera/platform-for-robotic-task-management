\documentclass[conference]{IEEEtran}
\ifCLASSINFOpdf
  \usepackage[pdftex]{graphicx}
\else
  \usepackage[dvips]{graphicx}
\fi

\usepackage[cmex10]{amsmath}
\usepackage{algorithmic}
\usepackage{algorithm}
\usepackage{array}

\ifCLASSOPTIONcompsoc
  \usepackage[caption=false,font=normalsize,labelfont=sf,textfont=sf]{subfig}
\else
  \usepackage[caption=false,font=footnotesize]{subfig}
\fi
\usepackage{fixltx2e}
\usepackage{stfloats}
\usepackage{url}
% correct bad hyphenation here
\hyphenation{op-tical net-works semi-conduc-tor}

\begin{document}
\title{Platform for Collaborative Robotics Task Management}

\author{\IEEEauthorblockN{Luke Fraser}
\IEEEauthorblockA{Computer Science Department\\
University of Nevada, Reno\\
Email: Luke.Fraser.A@gmail.com}}

\maketitle

\IEEEpeerreviewmaketitle
\begin{abstract}
Task planning in robotics is a difficult problem. The problem of task planning is known to be NP-hard. It is important for robots to be able to plan quickly in order for interaction with people to be fluid. These fast decisions enable the robot to work with people efficiently. A robot also needs to be able to understand the work done by others so that a collaborative work environment can be achieved. This report proposes a robot task architecture to achieve such collaborative behaviour with both multi-people and multi-robot tasks.
\end{abstract}

\section{Introduction}
As robots progress it is becoming more important for them to work side by side with people \cite{Breazeal2004}. I order for robots to work along side humans it is important that they are capable of understanding tasks as well as working towards an end goal.


\bibliographystyle{IEEEtran}
\bibliography{../refs/master.bib}
\end{document}